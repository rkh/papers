\documentclass{llncs}
\usepackage{makeidx}  % allows for indexgeneration
\usepackage[pdftex]{graphicx} % PNGs
\usepackage{amsmath, amssymb} % algebra
\usepackage[utf8x]{inputenc}
\usepackage[T1]{fontenc} 
\usepackage[procnames]{listings} % for sourcecode
\usepackage{graphviz} % graphs
\usepackage{array,multirow} % tables
\usepackage{afterpage} % figures
\usepackage{float} % figures

\lstset{%
	basicstyle=\small\ttfamily,
	language=Ruby,
	frame=lines,
	numbers=left,
	numberstyle=\rmfamily\tiny,
	numbersep=3pt,
	breaklines=true,
	breakatwhitespace=true
}

\restylefloat{figure}
\begin{document}
\pagestyle{headings}  % switches on printing of running heads
\mainmatter % start of the contributions
\title{Bithug - Social Coding}
\subtitle{A Code Repository Management Platform for Social Networks}
\titlerunning{Bithug}  % abbreviated title (for running head)
\author{Tim Felgentreff\and Konstantin Haase\and Johannes Wollert}
\date{\today}
\authorrunning{Felgentreff, Haase, Wollert}   % abbreviated author list (for running head)
\tocauthor{Tim Felgentreff (Hasso-Plattner-Institute)\and
	   Konstantin Haase (Hasso-Plattner-Institut)\and
	   Johannes Wollert (Hasso-Plattner-Institut)}
\institute{Social Web Applications Engineering, Internet Technologies and Systems, Hasso-Plattner-Institut, Universität Potsdam, D-14482 Potsdam, Germany,\\
\email{\{tim.felgentreff, konstantin.haase, johannes.wollert\}@student.hpi.uni-potsdam.de}}

\maketitle
\begin{abstract}

\end{abstract}
\section{Introduction}
\section{Project Management as a Service}
\subsection{Using Mixins for Software-Configuration}
\subsection{Scalability in Heterogeneous Environments}
\section{Social Networks as Organizing Bodies}
\subsection{The Lazy Web Concept}
\subsection{Decentralization of Responsibilities}
\subsection{Community-driven Innovation}
\end{document}
