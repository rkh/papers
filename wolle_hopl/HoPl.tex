\documentclass{llncs}
\usepackage{makeidx} % allows for indexgeneration
\usepackage[pdftex]{graphicx} % PNGs
\usepackage{amsmath, amssymb} % algebra
\usepackage[utf8x]{inputenc}
\usepackage[T1]{fontenc}
\usepackage[procnames]{listings} % for sourcecode
\usepackage{graphviz} % graphs
\usepackage{array,multirow} % tables
\usepackage{afterpage} % figures
\usepackage{float} % figures

\lstset{%
basicstyle=\small,
frame=single,
sensitive=true,
keywordsprefix=P,
keywords={END,W,W1},
keywordstyle=\bfseries,
identifierstyle=\ttfamily,
procnamestyle=\bfseries,
procnamekeys={P},
literate={<}{$<$}1 {>}{$>$}1 {=}{$=$}1 {<=}{$\leq$}1 {>=}{$\geq$}1 {=>}{{$\Rightarrow$}}1 {->}{{$\rightarrow$}}1
}

\restylefloat{figure}
\begin{document}
\pagestyle{headings} % switches on printing of running heads
\mainmatter % start of the contributions
\title{Erlang}
\subtitle{A concurrency-oriented Programming Language}
\titlerunning{Erlang} % abbreviated title (for running head)
\author{Johannes Wollert}
\date{\today}
\authorrunning{Johannes Wollert} % abbreviated author list (for running head)
\tocauthor{Johannes Wollert (Hasso-Plattner-Institute)}
\institute{History of Programming Languages, Software Architecture Group, Hasso-Plattner-Institut, Universit�t Potsdam, D-14482 Potsdam, Germany,\\
\email{johannes.wollert@student.hpi.uni-potsdam.de}}

\maketitle
\begin{abstract}
	In present times, concurrency is a matter of ever-growing importance, not only due to the movement towards multi-processor systems but also because of the inherently concurrent processes in our world. There are several models discribing possible solutions for the complex problems involved when handling largely concurrent and distributed systems such as scheduling, ressource management or communication between the multiple running processes or threads.
	The ''Erlang'' Programming Language implements an easy-to-use, yet powerful concept to tackle these challanges, providing an intuitive framework to build upon. 
\end{abstract}
\section{Introduction}
	Initially, Erlang was built for a single purpose alone, to significantly improve the development of telephony applications for Ericsson. To produce a language capable of manageing such systems, a large number of requirements need to be met. Of course, concurrency is one of them, but so is the ability to improve the system without even stopping it, a large amount of fault-tolerance and being able to control great numbers of processes, concurrently running in a distributed fashion. Also, telephony services need to be available all the time and because of that, the language must enable the designer to develop highly robust systems running ''forever''.
	Equipped with those clear goals, the idea started forming in 1985. In the following sections, we will outline the historical actions that lead to the formation of the Erlang language as we know it today. Later on, the language itself will be presented, covering syntax, types and paradigms.
\section{History}
  \subsection{The Need for Something new}
  	From 1974 until the end of the 1980's, the product generating most of Ericssons Profit was AXE, a telephony exchange written in PLEX, a proprietary language of Ericsson's used to specificly program the AXE hardware. It could not be used for any other devices. 
\section{The Language}
  \subsection{Paradigms}
  \subsection{Types}
  \subsection{Syntax}

\section{Conclusions}

\end{document}